
\chapter{LRC :Contrôle parité croisée}

    \section{Introduction}

        \paragraph{}
À partir de plusieurs message encodés grâce au VRC.
Pour rester cohérent avec notre message d'exemple nous prendrons le cas d'une matrice carrée composée de 7 messages de taille 7 chacun.
Il sera appliqué horizontalement à la matrice (à chaque message) le bit de parité.
La matrice passera donc à une taille de (7+1) sur 8.
Le dernier message qui sera généré grâce à l'application du bit de parité verticalement sur la matrice.


    \section{Fiabilité}

        \paragraph{}
%cb erreur peut être corrigée. ?
1 erreur peut être corrigée et jusqu’à trois erreurs peuvent être détectées à coup sûres.


    \section{Probabilité de détection}

        \paragraph{}
Probabilité d'exactitude d'un message :
\[  P(\text{Exact}) = P(\text{0 erreur}) + P(\text{1 erreur}) \]
\[ \text{Taille de la matrice } n = ( \text{ Nombre de lignes } + 1 ) * ( \text{ Longeur du message } + \text{ Bit de parité } ) \]
\[ P(\text{0 erreur}) + P(\text{1 erreur}) = (1-p)^{n} + {n\choose 1}*p*(1-p)^{n-1} \]
Dans notre cas et avec les 10\% de chance qu'un bit soit changé :
\[ \text{Taille de la matrice } n = 8*(7+1) = 8^{2} \]
\[ \text{Taille de la matrice } n = 64 \]
\[ P(\text{0 erreur}) + P(\text{1 erreur}) = (1-p)^{64} + 64*p*(1-p)^{63} \]
\[ P(\text{0 erreur}) + P(\text{1 erreur}) = (1-0.1)^{64} + 64*0.1*(1-0.1)^{63} \]
\[ P(\text{0 erreur}) + P(\text{1 erreur}) = (0.9)^{64} + 64*0.1*(0.9)^{63} \]
\[ P(\text{0 erreur}) + P(\text{1 erreur}) \approx 0.1179\% \]
Un message a donc environ 0.1179\% de chance d'être décodé correctement, c'est à dire non détecté faux.
Le nombre de cas d'erreur où le LRC est mis à défaut est :
\[  P(\text{Nombre d'indetections}) = (49+42+35+28+21+14+7)*7 \]
le nombre de cas total est  :
\[  P(\text{Nomber de cas}) = 2^{8*(7+1)} = 2^{8*8} \]
\[  P(\text{Nombre de cas}) = 2^{64} \]
\[  P(\text{Détection}) = \frac{P(\text{Nombre de cas}) - P(\text{Nombre d'indetections})}{P(\text{Nombre de cas})}*100 \]
\[  P(\text{Détection}) \approx 100\% \]


    \section{Rendement}

        \paragraph{}
Le rendement de ce code est bon :
\[  Rendement = \frac{n*\text{Taille du message}}{(n+1)*(\text{Taille du message}+1)} \]
        \paragraph{}
Dans le cas que nous étudions (7 messages de 7 bits):
\[  Rendement = \frac{7*7}{(7+1)*(7+1)} \approx 76.6\% \]
