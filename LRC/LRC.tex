
\chapter{LRC :Contrôle parité croisée}

    \section{Introduction}

        \paragraph{}
À partir de plusieurs message encodés grâce au VRC.
Pour rester cohérent avec notre message d'exemple nous prendrons le cas d'une matrice carrée composée de 7 messages de taille 7 chacun.
Il sera appliqué horizontalement à la matrice (à chaque message) le bit de parité.
La matrice passera donc à une taille de 7 sur 8.
Le dernier message qui sera généré grâce à l'application du bit de parité verticalement sur la matrice.


    \section{Fiabilité}

        \paragraph{}
%cb erreur peut être corrigée. ?


    \section{Probabilité de détection}

        \paragraph{}
%Probabilité de détecter une erreur :
%\[  P(\text{Détection}) = P(\text{1 erreur}) \]
%\[  P(\text{Détection}) = {n\choose k}p^k(1-p)^{n-k} \]
%\[  P(\text{Détection}) \approx 74\% \text{ avec } P(\text{Erreur}) = 10\% \]


    \section{Rendement}

        \paragraph{}
Le rendement de ce code est bon :
\[  Rendement = \frac{n*\text{Taille du message}}{(n+1)*(\text{Taille du message}+1)} \]
        \paragraph{}
Dans le cas que nous étudions (7 messages de 7 bits):
\[  Rendement = \frac{7*7}{(7+1)*(7+1)} \approx 76.6\% \]
