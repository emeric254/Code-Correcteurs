
\chapter{VRC : Bit de parité}

    \section{Introduction}

        \paragraph{}
le VRC (Vertical Redundancy Check), plus connu sous le nom de bit de parité,
est simplement le rajout d'un bit en fin de message pour assurer la parité du message.
Ce dernier bit la valeur nécessaire pour assurer un nombre pair de bit à 1 dans le message final.
Il est donc à 0 pour un nombre pair de bit à 1 dans le message de départ, ou est à 1.


    \section{Fiabilité}

        \paragraph{}
Une seule erreur peut être détectée à coup sur.
Toutes les erreurs où un nombre pair de bits sont modifiés ne sont pas détectées,
les erreurs détectées sont donc celles où un nombre impair de bits ont changé d'état.
Si une seule erreur intervient mais porte sur le bit de parité, le message est considéré comme invalide.
Ce code ne permet pas la correction d'erreur, il est nécessaire de redemander l'envoi du message détecté invalide.


    \section{Probabilité de détection}

        \paragraph{}
\[  P(\text{Transmission Parfaite}) = P(X=0) = {8\choose 0}p^0(1-p)^{8-0} = (1-p)^8 \]
        \paragraph{}
\[  P(\text{Message Erroné}) = 1 - P(\text{Transmission Parfaite}) = 1 - (1-p)^8 \]
        \paragraph{}
\[  P(\text{Détection}) = P(\text{1 erreur}) + P(\text{3 erreurs}) + P(\text{5 erreurs}) + P(\text{7 erreurs}) \]
\[ = {8\choose 1}p^1(1-p)^{8-1} + {8\choose 3}p^3(1-p)^{8-3} + {8\choose 5}p^5(1-p)^{8-5} + {8\choose 7}p^7(1-p)^{8-7} \]
\[ = 8p(1-p)^7 + {8\choose 3}p^3(1-p)^5 + {8\choose 5}p^5(1-p)^3 + 8p^7(1-p) \]
        \paragraph{}
\[  P(\text{Reconnaissance Erreur}) = \frac{P(\text{Détection})}{P(\text{Message Erroné})}\]
        \paragraph{}
Pour une probabilité de 10\% d'erreurs :
\[  P(\text{Transmission Parfaite}) = (1-0.1)^{8} = 43\%\]
\[  P(\text{Message Erroné}) = 1 - 0.43 = 57\% \]
\[  P(\text{Détection}) = 8*0.1(1-0.1)^7 + {8\choose 3}0.1^3(1-0.1)^5 + {8\choose 5}0.1^5(1-0.1)^3 + 8*0.1^7(1-0.1) \]
\[  P(\text{Détection}) = 0.8*0.9^7 + 56*0.1^3*0.9^5 + 56*0.1^5*0.9^3 + 8*0.1^7*0.9 \]
\[  P(\text{Détection}) = 42\% \]
\[  P(\text{Reconnaissance Erreur}) = \frac{0.42}{0.57} \approx 74\% \]


    \section{Rendement}

        \paragraph{}
Le rendement de ce code est très bon :
\[  Rendement = \frac{\text{Taille du message}}{\text{Taille du message}+1} \]
        \paragraph{}
Pour notre message d'exemple (un simple caractère encodé en UTF-7) le rendement est déjà excellent :
\[  Rendement = \frac{7}{7+1} = 87.5\% \]


    \section{Exemple pratique}
           \lstset{
                language=bash, basicstyle=\ttfamily\small, columns=flexible,
                tabsize=2, extendedchars=true, showspaces=false,
                showstringspaces=false, numbers=left, numberstyle=\tiny,
                breaklines=true, breakautoindent=true, captionpos=b
            }

    \begin{lstlisting}
>> perl envoiNoise.pl 9001 127.0.0.1:9000

 -> envoi du caractere : o

 -> nombre de caracteres envoyes : 100000

    \end{lstlisting}

    \begin{lstlisting}
>> perl receptionNoise.pl 9000 127.0.0.1:9001

 -> nombre de receptions : 100000
 -> nombre de receptions supposees bonnes : 58311
 -> nombre de vrais bons caracteres : 43079
 -> nombre d'erreurs au total : 56921
 -> nombre d'erreurs detectees : 41689
 -> nombre d'erreurs non detectees : 15232
 -> fiabilitee de l'envoi/reception : 43.079%
 -> fiabilitee de detection d'erreur : 73.2401047065231%

    \end{lstlisting}
