
\chapter{VRC : Bit de parité}

    \section{Introduction}

        \paragraph{}
le VRC (Vertical Redundancy Check), plus connu sous le nom de bit de parité,
est simplement le rajout d'un bit en fin de message pour assurer la parité du message.
Ce dernier bit la valeur nécessaire pour assurer un nombre pair de bit à 1 dans le message final.
Il est donc à 0 pour un nombre pair de bit à 1 dans le message de départ, ou est à 1.
        \paragraph{}
Une seule erreur peut être détectée à coup sur.
Cependant les erreurs détectées sont celles où un nombre impairs de bit ont changé d'état.


    \section{Probabilité de détection}

        \paragraph{}
Probabilité de détecter une erreur :
\[  P(D\acute{e}tection) = P(1 erreur) + P(3 erreurs) + P(5 erreurs) + P(7 erreurs) \]
\[  P(D\acute{e}tection) = {n\choose k}p^k(1-p)^{n-k} + {n\choose k} + {n\choose k} + {n\choose k} \]
\[  P(D\acute{e}tection) \approx 74\% \text{ avec } P( Erreur ) = 10\% \]


    \section{Rendement}

        \paragraph{}
Le rendement de ce code est très bon :
\[  Rendement = \frac{\text{Taille du message}}{\text{Taille du message}+1} \]
Pour un message sur 7 bits (un simple caractère encodé en UTF-7 par exemple) le rendement est déjà excellent :
\[  Rendement = \frac{7}{7+1} = 87.5\% \]
