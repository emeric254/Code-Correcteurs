
\chapter{CRC : Code de redondance cyclique}

    \section{Introduction}

        \paragraph{}
le CRC (Cyclic Redundancy Check), contrôle de redondance cyclique,
représente la principale méthode de détection d'erreurs utilisée dans les télécommunications et
consiste à protéger des blocs de données, appelés trames.
À chaque trame est associé un bloc de données, appelé code de contrôle (parfois CRC par abus de langage).
        \paragraph{}
On choisit un polynôme générateur, fixé et donc connu des deux entités qui se transmettent le message.
Grâce à celui ci, l'émetteur peut générer le code de contrôle qui est le reste de la division avec le message à envoyer.
Le récepteur divise ce qu'il a reçut, retrouve le message et sait si il y a eu un problème.
        \paragraph{}
Il existe plusieurs variantes du CRC selon le choix du polynôme : CRC 12, CRC 16, CRC CCIT v41, CRC 32, CRC ARPA.


    \clearpage

    \section{Fiabilité}

        \paragraph{}
Deux erreurs peuvent être détectées à coup sur grâce au CRC 16.
Les erreurs détectées sont seulement celles où un nombre impairs de bit ont changé d'état ou
celles qui sont des suites de bit qui ont tous changés (rafales), de taille inférieur au
degré du polynôme.
\\Un code polynomial C(k, n) permet de détecter toutes les
erreurs d’ordre $l \leq n-k$ (c’est-à-dire inférieur au degré du
polynôme générateur).
La probabilité de ne pas détecter les erreurs d’ordre $l>n-k$ est
très faible et égale à : $2-(n-k)$


    \section{Probabilité de détection}

        \paragraph{}
Nous sommes finalement arrivé jusqu'a ces calculs et n'avons pas reussi à en comprendre les secrets et malices
        \paragraph{}
Dans notre cas :
\[  \text{Polynôme CRC 16} = x^{16} + x^{15} + x^{2}+ 1 \]
%\[  P(\text{Message Erroné}) = 24bits \& P(\text{Erreur}) \]
%        \paragraph{}
%\[  P(\text{Détection}) = 1 - P(\text{Non Détection}) = 1 - [2-(n-k)] \]
%\[ = 1 - 2 + n - k = n - k - 1\]
%        \paragraph{}
%\[  P(\text{Reconnaissance Erreur}) = \frac{P(\text{Détection})}{P(\text{Message Erroné})}\]
%        \paragraph{}
%Pour une probabilité de 10\% d'erreurs et utilisation du CRC 16:
%\[  P(\text{Non Détection}) = 2-(n-k) \]
%\[  P(\text{Message Erroné}) = 24bits \& 10\% \]
%\[  P(\text{Détection}) = 8*0.1(1-0.1)^7 + {8\choose 3}0.1^3(1-0.1)^5 + {8\choose 5}0.1^5(1-0.1)^3 + 8*0.1^7(1-0.1) \]
%\[  P(\text{Détection}) = 0.8*0.9^7 + 56*0.1^3*0.9^5 + 56*0.1^5*0.9^3 + 8*0.1^7*0.9 \]
%\[  P(\text{Détection}) = 42\% \]
%\[  P(\text{Reconnaissance Erreur}) = \frac{0.42}{0.57} \approx 74\% \]
% vers 99% apparement ;)


    \section{Rendement}

        \paragraph{}
Le rendement de ce code est dépendant de la taille du message :
\[  Rendement = \frac{\text{Taille du message}}{\text{Taille du message} + \text{Degré du polynôme}} \]
        \paragraph{}
Pour notre message d'exemple, un message de seulement 7 bit de longueur, un codage avec du CRC 16 donne un rendement très médiocre :
\[  Rendement = \frac{7}{7 + 16} \approx 30\%  \]
        \paragraph{}
Cependant avec un message de taille plus importante comme par exemple un message de 128 bit avec du CRC 16 le rendement devient excellent :
\[  Rendement = \frac{128}{128 + 16} \approx 89\%  \]


    \section{Exemple de calcul}
        \paragraph{Codage}
Prenons comme exemple un message à envoyer de valeur binaire $ 0011001 $
et avec le polynôme $ \text{CRC 16} = 1100000000000101 $.
\\On décale le message de 16 bits vers la gauche $ 00110010000000000000000 $.
Ensuite on calcul la somme de contrôle (checksum) de ce message
\[ 00110010000000000000000 | 1100000000000101 \]
\[ xor 110000000000010100000 \]
\[ = 000010000000010100000 | 1100000000000101 \]
\[ xor 11000000000001010 \]
\[ = 01000000010101010 | 1100000000000101 \]
\[ xor 1100000000000101 \]
\[ = 0100000010101111 \]
La somme de contrôle est donc :
\[ 0100000010101111 \]
On concatène cela au message originale à envoyer pour obtenir le message à envoyer :
\[ 00110010100000010101111 \]
        \paragraph{Décodage}
On recois un message de valeur binaire $ 00110010100000010101111 $
et avec le polynôme $ \text{CRC 16} = 1100000000000101 $.
On divise alors ce message par le polynôme pour le vérifier
\[ 00110010100000010101111 | 1100000000000101 \]
\[ xor 110000000000010100000 \]
\[ = 000010100000000001111 | 1100000000000101 \]
\[ xor 11000000000001010 \]
\[ = 01100000000000101 | 1100000000000101 \]
\[ xor 1100000000000101 \]
\[ = 0000000000000000 \]
On s'apercoit que le résultat est nul, aucune erreur n'a été détectée.
\\On récupère le message en décalant ce que l'on a reçu de 16 bits à droite $ 0011001 $

    \section{Exemple pratique}
           \lstset{
                language=bash, basicstyle=\ttfamily\small, columns=flexible,
                tabsize=2, extendedchars=true, showspaces=false,
                showstringspaces=false, numbers=left, numberstyle=\tiny,
                breaklines=true, breakautoindent=true, captionpos=b
            }
Voir détail du code en annexe.
    \begin{lstlisting}
>> perl envoiNoise.pl 9001 127.0.0.1:9000

 -> envoi du caractere : o

 -> nombre de caracteres envoyes : 100000

    \end{lstlisting}

    \begin{lstlisting}
>> perl receptionNoise.pl 9000 127.0.0.1:9001

 -> nombre de receptions : 100000
 -> nombre de receptions supposees bonnes : 7557
 -> nombre de vrais bons caracteres : 7301
 -> nombre d'erreurs au total : 92699
 -> nombre d'erreurs detectees : 92443
 -> nombre d'erreurs non detectees : 256
 -> fiabilitee de l'envoi/reception : 7.301%
 -> fiabilitee de detection d'erreur : 99.7238373660989%

    \end{lstlisting}
