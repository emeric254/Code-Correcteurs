\documentclass[a4paper,11pt]{report}
\usepackage[utf8]{inputenc}
\usepackage{color,amsmath,xcolor,listings,graphicx}
\usepackage[francais]{babel}

% paramétrage pour les zones de code perl
\lstset{
    language=Perl, commentstyle=\textit, frame=shadowbox,
    rulesepcolor=\color{gray}, basicstyle=\ttfamily\small, columns=flexible,
    tabsize=3, extendedchars=true, showspaces=false,
    showstringspaces=false, numbers=left, numberstyle=\tiny,
    breaklines=true, breakautoindent=true, captionpos=b, morecomment=[l]{//}
}

%language=Octave %-> choose the language of the code
%basicstyle=\footnotesize %-> the size of the fonts used for the code
%numbers=left %-> where to put the line-numbers
%numberstyle=\footnotesize %-> size of the fonts used for the line-numbers
%stepnumber=2 -> the step between two line-numbers.
%numbersep=5pt -> how far the line-numbers are from the code
%backgroundcolor=\color{white} -> sets background color (needs package)
%showspaces=false -> show spaces adding particular underscores
%showstringspaces=false -> underline spaces within strings
%showtabs=false -> show tabs within strings through particular underscores
%frame=single -> adds a frame around the code
%tabsize=2 -> sets default tab-size to 2 spaces
%captionpos=b -> sets the caption-position to bottom
%breaklines=true -> sets automatic line breaking
%breakatwhitespace=false -> automatic breaks happen at whitespace
%morecomment=[l]{//} -> displays comments in italics (language dependent)


% infos du document
\title{Codes Correcteurs}
\author{Julie Badets, Corentin Frade, Quentin Rouland, Émeric Tosi}
\date{\today}


\begin{document}

    \maketitle{} % Afficher la page de garde : Titre + Auteur(s) + Date de dernière compilation


%    \begin{figure} % on s'en fout de l'image moche xD
%        \begin{center}
            %\includegraphics{network.png}
            %\includegraphics[height=128, width=128]{network.png}
            %\includegraphics[scale=0.5]{network.png}
%        \end{center}
%            \caption{ Laule } % ce qui apparait juste en dessous de l'image
%            \label{c'est styler !}
%    \end{figure}


    \begin{abstract} % Introduction // Résumé
        \paragraph{}
Un code correcteur d'erreur est utilisé pour transmettre un message dans un canal bruité ;
il permet de reconstituer le message émis même si des erreurs (en nombre limité), dues au bruit, ont altéré le message.
L'alphabet source, comme l'alphabet du code, est $\{0,1\}$ .
On s'intéresse au codage par blocs : chaque mot de longueur $ m $ est codé par un mot de longueur $n$ avec $n \geq m$.
Le codage est donc une application de $\{0,1\}^{m}$  vers $\{0,1\}^{n}$.
Parmi les $n$ bits du mot-code que nous allons décrire, $m$ reproduisent le mot-source,
les $n-m$ autres sont les bits de correction : le taux de transmission est de $\frac{n}{m}$.
        \paragraph{}
On considère les erreurs comme indépendantes les unes des autres et tous les bits ont la même probabilité d'erreur.
Nous nous intéressons donc aux codes correcteurs d'une façon plutôt théorique.
En pratique, si on prend un exemple dans les communications sans fil,
des problèmes de parasites se posent et l'indépendance des erreurs est compromise.
        \paragraph{}
Pour la suite nous prendrons comme exemple un message qui est un simple caractère encodé en UTF-7.
Ce message a donc une taille de 7 bits.
Cela égalise ainsi les calculs et l'implémentation pour les tests.
    \end{abstract}
    \clearpage


    \setcounter{tocdepth}{1} % définir la profondeur de l'Index
    \renewcommand{\contentsname}{Sommaire} % renommer l'Index en Sommaire
    \tableofcontents{} % afficher l'Index
    \clearpage


% Differentes Parties / Chapitres / Autres fichiers à inclure :

    
\chapter{Code de répétition}

    \section{Introduction}

        \paragraph{}
On transmet simplement plusieurs fois le même message.
        \paragraph{}
Une seule erreur peut être détectée à coup sur.


    \section{Probabilité de détection}

        \paragraph{}
Probabilité de détecter une erreur :
%\[  P(D\acute{e}tection) = P(1 erreur) + P(3 erreurs) + P(5 erreurs) + P(7 erreurs) \]
%\[  P(D\acute{e}tection) = {n\choose k}p^k(1-p)^{n-k} + {n\choose k} + {n\choose k} + {n\choose k} \]
%\[  P(D\acute{e}tection) \approx 74\% \text{ avec } P( Erreur ) = 10\% \]


    \section{Rendement}

        \paragraph{}
Le rendement de ce code est très mauvais, on double au minimum la taille du message :
\[  Rendement = \text{Taille du message}*\text{Nombre de répétition(s)} \]
Pour un message sur 7 bits (un simple caractère encodé en UTF-7 par exemple) et avec 1 répétition seulement :
\[  Rendement = \frac{7}{7*2} = 50\% \]
Pour le même message mais avec 2 répétitions :
\[  Rendement = \frac{7}{7*3} \approx 33.3\% \]


    \section{Implémentation}

        \paragraph{}

    \clearpage

    
\chapter{VRC : Bit de parité}

    \section{Introduction}

        \paragraph{}
le VRC (Vertical Redundancy Check), plus connu sous le nom de bit de parité,
est simplement le rajout d'un bit en fin de message pour assurer la parité du message.
Ce dernier bit la valeur nécessaire pour assurer un nombre pair de bit à 1 dans le message final.
Il est donc à 0 pour un nombre pair de bit à 1 dans le message de départ, ou est à 1.
        \paragraph{}
Une seule erreur peut être détectée à coup sur.
Cependant les erreurs détectées sont celles où un nombre impairs de bit ont changé d'état.


    \section{Probabilité de détection}

        \paragraph{}
Probabilité de détecter une erreur :
\[  P(D\acute{e}tection) = P(1 erreur) + P(3 erreurs) + P(5 erreurs) + P(7 erreurs) \]
\[  P(D\acute{e}tection) = {n\choose k}p^k(1-p)^{n-k} + {n\choose k} + {n\choose k} + {n\choose k} \]
\[  P(D\acute{e}tection) \approx 74\% \text{ avec } P( Erreur ) = 10\% \]


    \section{Rendement}

        \paragraph{}
Le rendement de ce code est très bon :
\[  Rendement = \frac{\text{Taille du message}}{\text{Taille du message}+1} \]
Pour un message sur 7 bits (un simple caractère encodé en UTF-7 par exemple) le rendement est déjà excellent :
\[  Rendement = \frac{7}{7+1} = 87.5\% \]

    \clearpage

    
\chapter{LRC :Contrôle parité croisée}

    \section{Introduction}

        \paragraph{}
À partir de plusieurs message encodés grâce au VRC.
Nous prendrons le cas d'une matrice carrée composée de 7 messages de taille 7 chacun.
Il sera appliqué horizontalement à la matrice (à chaque message) le bit de parité.
La matrice passera donc à une taille de 7 sur 8.
Le dernier message qui sera généré grâce à l'application du bit de parité verticalement sur la matrice.
        \paragraph{}
cb erreur peut être corrigée. ?


    \section{Probabilité de détection}

        \paragraph{}
Probabilité de détecter une erreur :
%\[  P(D\acute{e}tection) = P(1 erreur) + P(3 erreurs) + P(5 erreurs) + P(7 erreurs) \]
%\[  P(D\acute{e}tection) = {n\choose k}p^k(1-p)^{n-k} + {n\choose k} + {n\choose k} + {n\choose k} \]
%\[  P(D\acute{e}tection) \approx 74\% \text{ avec } P( Erreur ) = 10\% \]


    \section{Rendement}

        \paragraph{}
Le rendement de ce code est bon :
\[  Rendement = \frac{n*\text{Taille du message}}{(n+1)*(\text{Taille du message}+1)} \]
Dans le cas que nous étudions (7 messages de 7 bits):
\[  Rendement = \frac{7*7}{(7+1)*(7+1)} \approx 76.6\% \]

    \clearpage

    
\chapter{CRC : Code de redondance cyclique}

    \section{Introduction}

        \paragraph{}
le CRC (Cyclic Redundancy Check), contrôle de redondance cyclique,
représente la principale méthode de détection d'erreurs utilisée dans les télécommunications et
consiste à protéger des blocs de données, appelés trames.
À chaque trame est associé un bloc de données, appelé code de contrôle (parfois CRC par abus de langage).
        \paragraph{}
On choisit un polynôme générateur, fixé et donc connu des deux entités qui se transmettent le message.
Grâce à celui ci, l'émetteur peut générer le code de contrôle qui est le reste de la division avec le message à envoyer.
Le récepteur divise ce qu'il a reçut, retrouve le message et sait si il y a eu un problème.
        \paragraph{}
Il existe plusieurs variantes du CRC selon le choix du polynôme : CRC 12, CRC 16, CRC CCIT v41, CRC 32, CRC ARPA.


    \clearpage

    \section{Fiabilité}

        \paragraph{}
Deux erreurs peuvent être détectées à coup sur grâce au CRC 16.
Les erreurs détectées sont seulement celles où un nombre impairs de bit ont changé d'état ou
celles qui sont des suites de bit qui ont tous changés (rafales), de taille inférieur au
degré du polynôme.
\\Un code polynomial C(k, n) permet de détecter toutes les
erreurs d’ordre $l \leq n-k$ (c’est-à-dire inférieur au degré du
polynôme générateur).
La probabilité de ne pas détecter les erreurs d’ordre $l>n-k$ est
très faible et égale à : $2-(n-k)$


    \section{Probabilité de détection}

        \paragraph{}
Nous sommes finalement arrivé jusqu'a ces calculs et n'avons pas reussi à en comprendre les secrets et malices
        \paragraph{}
Dans notre cas :
\[  \text{Polynôme CRC 16} = x^{16} + x^{15} + x^{2}+ 1 \]
%\[  P(\text{Message Erroné}) = 24bits \& P(\text{Erreur}) \]
%        \paragraph{}
%\[  P(\text{Détection}) = 1 - P(\text{Non Détection}) = 1 - [2-(n-k)] \]
%\[ = 1 - 2 + n - k = n - k - 1\]
%        \paragraph{}
%\[  P(\text{Reconnaissance Erreur}) = \frac{P(\text{Détection})}{P(\text{Message Erroné})}\]
%        \paragraph{}
%Pour une probabilité de 10\% d'erreurs et utilisation du CRC 16:
%\[  P(\text{Non Détection}) = 2-(n-k) \]
%\[  P(\text{Message Erroné}) = 24bits \& 10\% \]
%\[  P(\text{Détection}) = 8*0.1(1-0.1)^7 + {8\choose 3}0.1^3(1-0.1)^5 + {8\choose 5}0.1^5(1-0.1)^3 + 8*0.1^7(1-0.1) \]
%\[  P(\text{Détection}) = 0.8*0.9^7 + 56*0.1^3*0.9^5 + 56*0.1^5*0.9^3 + 8*0.1^7*0.9 \]
%\[  P(\text{Détection}) = 42\% \]
%\[  P(\text{Reconnaissance Erreur}) = \frac{0.42}{0.57} \approx 74\% \]
% vers 99% apparement ;)


    \section{Rendement}

        \paragraph{}
Le rendement de ce code est dépendant de la taille du message :
\[  Rendement = \frac{\text{Taille du message}}{\text{Taille du message} + \text{Degré du polynôme}} \]
        \paragraph{}
Pour notre message d'exemple, un message de seulement 7 bit de longueur, un codage avec du CRC 16 donne un rendement très médiocre :
\[  Rendement = \frac{7}{7 + 16} \approx 30\%  \]
        \paragraph{}
Cependant avec un message de taille plus importante comme par exemple un message de 128 bit avec du CRC 16 le rendement devient excellent :
\[  Rendement = \frac{128}{128 + 16} \approx 89\%  \]


    \section{Exemple de calcul}
        \paragraph{Codage}
Prenons comme exemple un message à envoyer de valeur binaire $ 0011001 $
et avec le polynôme $ \text{CRC 16} = 1100000000000101 $.
\\On décale le message de 16 bits vers la gauche $ 00110010000000000000000 $.
Ensuite on calcul la somme de contrôle (checksum) de ce message
\[ 00110010000000000000000 | 1100000000000101 \]
\[ xor 110000000000010100000 \]
\[ = 000010000000010100000 | 1100000000000101 \]
\[ xor 11000000000001010 \]
\[ = 01000000010101010 | 1100000000000101 \]
\[ xor 1100000000000101 \]
\[ = 0100000010101111 \]
La somme de contrôle est donc :
\[ 0100000010101111 \]
On concatène cela au message originale à envoyer pour obtenir le message à envoyer :
\[ 00110010100000010101111 \]
        \paragraph{Décodage}
On recois un message de valeur binaire $ 00110010100000010101111 $
et avec le polynôme $ \text{CRC 16} = 1100000000000101 $.
On divise alors ce message par le polynôme pour le vérifier
\[ 00110010100000010101111 | 1100000000000101 \]
\[ xor 110000000000010100000 \]
\[ = 000010100000000001111 | 1100000000000101 \]
\[ xor 11000000000001010 \]
\[ = 01100000000000101 | 1100000000000101 \]
\[ xor 1100000000000101 \]
\[ = 0000000000000000 \]
On s'apercoit que le résultat est nul, aucune erreur n'a été détectée.
\\On récupère le message en décalant ce que l'on a reçu de 16 bits à droite $ 0011001 $

    \section{Exemple pratique}
           \lstset{
                language=bash, basicstyle=\ttfamily\small, columns=flexible,
                tabsize=2, extendedchars=true, showspaces=false,
                showstringspaces=false, numbers=left, numberstyle=\tiny,
                breaklines=true, breakautoindent=true, captionpos=b
            }
Voir détail du code en annexe 2.
    \begin{lstlisting}
>> perl envoiNoise.pl 9001 127.0.0.1:9000

 -> envoi du caractere : o

 -> nombre de caracteres envoyes : 100000

    \end{lstlisting}

    \begin{lstlisting}
>> perl receptionNoise.pl 9000 127.0.0.1:9001

 -> nombre de receptions : 100000
 -> nombre de receptions supposees bonnes : 7557
 -> nombre de vrais bons caracteres : 7301
 -> nombre d'erreurs au total : 92699
 -> nombre d'erreurs detectees : 92443
 -> nombre d'erreurs non detectees : 256
 -> fiabilitee de l'envoi/reception : 7.301%
 -> fiabilitee de detection d'erreur : 99.7238373660989%

    \end{lstlisting}

    \clearpage

    \chapter{Conclusion}
Nous sommes simplement mauvais parce que nous le valons bien ;)

    
\appendix

    \section{Exemple(s)}

        \paragraph{}
            \emph{emphatique}
            \textbf{gras}
            \texttt{machine à écrire}
            \textsl{incliné}
            \textsc{Petites majuscules}

        \paragraph{}
            The foundations of the rigorous study of \emph{analysis}
            were laid in the nineteenth century, notably by the
            mathematicians Cauchy and Weierstrass. Central to the
            study of this subject are the formal definitions of
            \emph{limits} and \emph{continuity}.

        \paragraph{}
            Let $D$ be a subset of $\bf R$ and let
            $f \colon D \to \mathbf{R}$ be a real-valued function on
            $D$. The function $f$ is said to be \emph{continuous} on
            $D$ if, for all $\epsilon > 0$ and for all $x \in D$,
            there exists some $\delta > 0$ (which may depend on $x$)
            such that if $y \in D$ satisfies
            \[ |y - x| < \delta \]
            then
            \[ |f(y) - f(x)| < \epsilon. \]

        \paragraph{}
            One may readily verify that if $f$ and $g$ are continuous
            functions on $D$ then the functions $f+g$, $f-g$ and
            $f.g$ are continuous. If in addition $g$ is everywhere
            non-zero then $f/g$ is continuous.



    %%configuration de listings
    \lstset{
        language=perl,
        basicstyle=\ttfamily\small, %
        columns=flexible, %
        tabsize=2, %
        extendedchars=true, %
        showspaces=false, %
        showstringspaces=false, %
        numbers=left, %
        numberstyle=\tiny, %
        breaklines=true, %
        breakautoindent=true, %
        captionpos=b
    }

%language=Octave -> choose the language of the code
%basicstyle=\footnotesize -> the size of the fonts used for the code
%numbers=left -> where to put the line-numbers
%numberstyle=\footnotesize -> size of the fonts used for the line-numbers
%stepnumber=2 -> the step between two line-numbers.
%numbersep=5pt -> how far the line-numbers are from the code
%backgroundcolor=\color{white} -> sets background color (needs package)
%showspaces=false -> show spaces adding particular underscores
%showstringspaces=false -> underline spaces within strings
%showtabs=false -> show tabs within strings through particular underscores
%frame=single -> adds a frame around the code
%tabsize=2 -> sets default tab-size to 2 spaces
%captionpos=b -> sets the caption-position to bottom
%breaklines=true -> sets automatic line breaking
%breakatwhitespace=false -> automatic breaks happen at whitespace
%morecomment=[l]{//} -> displays comments in italics (language dependent)


    \begin{lstlisting}[numbers=left, breaklines=true]
        my $url = 'http://192.168.0.7:8080/POG/getINDI.jsp?zarefno=a1.';
        use LWP::Simple;
        my $content = get $url;
        die "Couldn't get $url" unless defined $content;
        print $content;
        print "\n";
        print "Length " + length($content)
    \end{lstlisting}

\lstinputlisting{./VRC/Perl/receptionNoise.pl}

\lstinputlisting{./VRC/Perl/envoiNoise.pl}

%    \listoffigures % index des images du rapport

\end{document}
