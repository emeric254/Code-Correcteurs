
\chapter{Code de répétition}

    \section{Introduction}

        \paragraph{}
On transmet simplement plusieurs fois le même message, on double les bits à transmettre pour recevoir deux fois le même message,
si les messages reçus ne concordent pas, alors il y a eu une erreur dans la transmission.
On peut choisir deux méthodes pour doubler les bits, doubler chaque bit ou doubler le message complet.
\\ Exemple :
\\ 101 deviens 110011 si on double chaque bit
\\ 101 deviens 101101 si on double le message


    \clearpage

    \section{Fiabilité}

        \paragraph{}
Une seule erreur peut être détectée à coup sur.
Les erreurs ne sont pas detectés dans les cas d'un nombre d'erreur pair se situant au même bits dans les deux messages envoyés.


    \section{Probabilité de détection}

        \paragraph{}
Nombre d'erreurs qui ne sont pas détectées (par dénombrement des cas d'indection cités précédement):
\[  \text{Nombre d'erreurs indétectables} = 123 \]
Nombre de cas totaux :
\[  \text{Nombre de cas} = 2^{(7*2)} \]
Probabilité de détection :
\[  P(\text{Détection}) = \frac{\text{Nombre de cas} - \text{Nombre d'erreurs indétectables}}{\text{Nombre de cas}}*100 \]
On obtient alors :
\[  P(\text{Détection}) = \frac{2^{(7*2)} - 123}{2^{(7*2)}} \]
\[  P(\text{Détection}) \approx 99.25\% \]


    \section{Rendement}

        \paragraph{}
Le rendement de ce code est très mauvais, on double au minimum la taille du message :
\[  Rendement = \text{Taille du message}*\text{Nombre de répétitions} \]
Pour un message sur 7 bits (un simple caractère encodé en UTF-7 par exemple) et avec 1 répétition seulement :
\[  Rendement = \frac{7}{7*2} = 50\% \]
Pour le même message mais avec 2 répétitions :
\[  Rendement = \frac{7}{7*3} \approx 33.3\% \]

    \section{Exemple pratique}

        \lstset{
            language=bash, basicstyle=\ttfamily\small, columns=flexible,
            tabsize=2, extendedchars=true, showspaces=false,
            showstringspaces=false, numbers=left, numberstyle=\tiny,
            breaklines=true, breakautoindent=true, captionpos=b
        }
Voir détail du code en annexe.
        \begin{lstlisting}
>> perl envoiNoise.pl 8000 127.0.0.1:9000

 -> envoi du caractere : o 
 
 -> nombre de caracteres envoyes : 100000

        \end{lstlisting}

        \begin{lstlisting}

>> perl receptionNoise.pl 9000 127.0.0.1:8000

 -> nombre de receptions : 100000
 -> nombre de receptions supposees bonnes : 18867
 -> nombre de vrais bons caracteres : 18255
 -> nombre d'erreurs au total : 81745
 -> nombre d'erreurs detectees : 81133
 -> nombre d'erreurs non detectees : 612
 -> fiabilitee de l'envoi/reception : 18.255%
 -> fiabilitee de detection d'erreur : 99.2513303565967%
 
        \end{lstlisting}
