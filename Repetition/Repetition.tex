
\chapter{Code de répétition}

    \section{Introduction}

        \paragraph{}
On transmet simplement plusieurs fois le même message.


    \section{Fiabilité}

        \paragraph{}
Une seule erreur peut être détectée à coup sur.


    \section{Probabilité de détection}

        \paragraph{}
%Probabilité de détecter une erreur :
%\[  P(\text{Détection}) = P(\text{1 erreur}) \]
%\[  P(\text{Détection}) = {n\choose k}p^k(1-p)^{n-k} \]
%\[  P(\text{Détection}) \approx 74\% \text{ avec } P(\text{Erreur}) = 10\% \]


    \section{Rendement}

        \paragraph{}
Le rendement de ce code est très mauvais, on double au minimum la taille du message :
\[  Rendement = \text{Taille du message}*\text{Nombre de répétitions} \]
Pour un message sur 7 bits (un simple caractère encodé en UTF-7 par exemple) et avec 1 répétition seulement :
\[  Rendement = \frac{7}{7*2} = 50\% \]
Pour le même message mais avec 2 répétitions :
\[  Rendement = \frac{7}{7*3} \approx 33.3\% \]
