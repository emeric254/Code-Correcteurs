
\chapter{Code de répétition}

    \section{Introduction}

        \paragraph{}
On transmet simplement plusieurs fois le même message, on double les bits à transmettre pour recevoir deux fois le même message, si les messages reçu ne concordent pas, alors il y a eu une erreur dans la transmission. On peu choisir deux méthodes pour doubler les bits, doubler chaque bit ou doubler le message complet. Exemple :

    101 deviens 110011 si on double chaque bit
    101 deviens 101101 si on double le message



    \section{Fiabilité}

        \paragraph{}
Une seule erreur peut être détectée à coup sur.


    \section{Probabilité de détection}

        \paragraph{}
Nombre d'erreurs qui ne sont pas détectées (par dénombrement):
\[  \text{Nombre d'erreurs indétectables} = 123 \]
Nombre de cas totaux :
\[  \text{Nombre de cas} = 2^{(7*2)} \]
Probabilité de détection :
\[  P(\text{Détection}) = \frac{\text{Nombre de cas} - \text{Nombre d'erreurs indétectables}}{\text{Nombre de cas}}*100 \]
On obtient alors :
\[  P(\text{Détection}) = \frac{2^{(7*2)} - 123}{2^{(7*2)}} \]
\[  P(\text{Détection}) \approx 99.25\% \]


    \section{Rendement}

        \paragraph{}
Le rendement de ce code est très mauvais, on double au minimum la taille du message :
\[  Rendement = \text{Taille du message}*\text{Nombre de répétitions} \]
Pour un message sur 7 bits (un simple caractère encodé en UTF-7 par exemple) et avec 1 répétition seulement :
\[  Rendement = \frac{7}{7*2} = 50\% \]
Pour le même message mais avec 2 répétitions :
\[  Rendement = \frac{7}{7*3} \approx 33.3\% \]
