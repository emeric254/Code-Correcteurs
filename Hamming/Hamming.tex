
\chapter{Hamming}

    \section{Introduction}

        \paragraph{}
On montre que si deux mots distincts du code de hamming diffèrent au moins en d bits,
alors le code permet de corriger exactement (d-1)/2 erreurs.


    \section{Fiabilité}

        \paragraph{}
%cb erreur peut être corrigée. ?


    \section{Probabilité de détection}

        \paragraph{}
%Probabilité de détecter une erreur :
%\[  P(\text{Détection}) = P(\text{1 erreur}) \]
%\[  P(\text{Détection}) = {n\choose k}p^k(1-p)^{n-k} \]
%\[  P(\text{Détection}) \approx 74\% \text{ avec } P(\text{Erreur}) = 10\% \]


    \section{Rendement}

        \paragraph{}
%Le rendement de ce code est bon :
%\[  Rendement = \frac{n*\text{Taille du message}}{(n+1)*(\text{Taille du message}+1)} \]
%Dans le cas que nous étudions (7 messages de 7 bits):
%\[  Rendement = \frac{7*7}{(7+1)*(7+1)} \approx 76.6\% \]
