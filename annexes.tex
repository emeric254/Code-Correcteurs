
\appendix{}

\chapter{Implémentations}

    %%configuration de listings
    \lstset{
        language=perl, basicstyle=\ttfamily\small, columns=flexible,
        tabsize=2, extendedchars=true, showspaces=false,
        showstringspaces=false, numbers=left, numberstyle=\tiny,
        breaklines=true, breakautoindent=true, captionpos=b
    }

%language=Octave -> choose the language of the code
%basicstyle=\footnotesize -> the size of the fonts used for the code
%numbers=left -> where to put the line-numbers
%numberstyle=\footnotesize -> size of the fonts used for the line-numbers
%stepnumber=2 -> the step between two line-numbers.
%numbersep=5pt -> how far the line-numbers are from the code
%backgroundcolor=\color{white} -> sets background color (needs package)
%showspaces=false -> show spaces adding particular underscores
%showstringspaces=false -> underline spaces within strings
%showtabs=false -> show tabs within strings through particular underscores
%frame=single -> adds a frame around the code
%tabsize=2 -> sets default tab-size to 2 spaces
%captionpos=b -> sets the caption-position to bottom
%breaklines=true -> sets automatic line breaking
%breakatwhitespace=false -> automatic breaks happen at whitespace
%morecomment=[l]{//} -> displays comments in italics (language dependent)


    %\section{Exemple(s)}

        %\paragraph{}
            %\emph{emphatique}
            %\textbf{gras}
            %\texttt{machine à écrire}
            %\textsl{incliné}
            %\textsc{Petites majuscules}

        %\paragraph{}
            %The foundations of the rigorous study of \emph{analysis}
            %were laid in the nineteenth century, notably by the
            %mathematicians Cauchy and Weierstrass. Central to the
            %study of this subject are the formal definitions of
            %\emph{limits} and \emph{continuity}.

        %\paragraph{}
            %Let $D$ be a subset of $\bf R$ and let
            %$f \colon D \to \mathbf{R}$ be a real-valued function on
            %$D$. The function $f$ is said to be \emph{continuous} on
            %$D$ if, for all $\epsilon > 0$ and for all $x \in D$,
            %there exists some $\delta > 0$ (which may depend on $x$)
            %such that if $y \in D$ satisfies
            %\[ |y - x| < \delta \]
            %then
            %\[ |f(y) - f(x)| < \epsilon. \]

        %\paragraph{}
            %One may readily verify that if $f$ and $g$ are continuous
            %functions on $D$ then the functions $f+g$, $f-g$ and
            %$f.g$ are continuous. If in addition $g$ is everywhere
            %non-zero then $f/g$ is continuous.


    \section{VRC}

        \paragraph{}
Implémentation de l'envoi pour le VRC
\lstinputlisting{./VRC/Perl/receptionNoise.pl}

        \paragraph{}
Implémentation de la reception pour le VRC
\lstinputlisting{./VRC/Perl/envoiNoise.pl}

        \paragraph{}
Implémentation du VRC
\lstinputlisting{./VRC/Perl/Parity.pm}
